\section{Project Conclusion}

In this project, we evaluated the relation between IO-depth, latency
and bandwidth.  As the IO-depth grows, the latency of read/write
operations also grows linearly, but the bandwidth will peak as the
device becomes saturated at a certain IO-depth and this depth is
variable. In order to archive the best performance, we proposed a new
approach called SFQ($D^2$) that benchmarks the device and sets a
target latency as parameter which is used to tune IO-depth dynamically
in real-time. To garuntee fairness, we also adopt the SFQ algorithm to
sort requests at dispatch.

From the evaluation results comparing the noop and SFQD scheduler in
different IO-depth, SFQ($D^2$) maintains high bandwidth and lower
latency compare with sfqd in different IO-depth and the same
performance as noop. And in competing workloads, SFQ($D^2$) is more
fair than other schedulers.

\section{Future Works}

The SFQ($D^2$) scheduler performance is better than SFQD but it still
has many points which can be improved. The improvements include both
implementation and evaluation.

The calculation for average latency should separate the read and write
operations. That means we should calculate the average read/write
latency and assigned the IO-depth base on their shared in a time
window.  Not just calculate a total latency and divide them with
read/write operation percentage.

For the evaluation, we need to implement FlashFQ scheduler to do
comparison. Furthermore, in competing workloads, we need to evaluate
the situation where both intensive and sparse processes are read or
write, then compare with the other schedulers.

\section{Sources}

The git repository for this project is publically accessable at:


\centering \verb+https://github.com/BigTheta/SFQ-D2-.git+

