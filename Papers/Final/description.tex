\section{Project Description}

SFQ, SFQ(D), and SFQ($D^2$) were implemented as linux I/O schedulers
on kernel 3.7.0. The bulk of implementation lies in the basic SFQ so
we will explain in this detail first, then elaborate on the
differences between the later implementations. 

\subsection{SFQ}

SFQ is a simple start-time fair queueing scheduler which dispatches
requests based on their ``start tag'' value, which is, essentially, a
logical timestamp representing the virtual time the request was
recieved by the scheduler (initiated by the \emph{request\_add}
function. Additionally, each request is assigned a ``finish tag'',
which is the sum of the start tag and some cost assosicated with the
function. In our implementation, the cost is simply the size of the
request in blocks. The virtual time of the system can be determined
through several methods, depending on implementation. Each
implementatino has implications outside of the scope of this
paper. Suffice to say, our implementation uses MIN-SFQ, which say
virtual time shoudl be determined by the minimum start tag of all
outstanding requests (includes the queues and requested which have
been dispatched but not completed). Each process also has its own
virtual time, which is the finish tag assigned to the last request
issued by that process. When a new request arrives, its start tag is
set to the maximum of its process' virtual time and system virtual
time. For simplicity, a request queue is maintaiend for each process,
so new requests can easily be inserted at the tail of their respective
queue.

In order to implement this design, SFQ has to allocate an in memory
structure for each process attempting to do I/O. During calls to
\emph{set\_request}, SFQ must: 

\begin{enumerate}
\item
  Identify which process a request belongs to.
\item
  Determine if that process already has a representation in the scheduler.
\item
  If not, create one.
\item
  Attach a pointer to that process' structure to the request's private field.
\end{enumerate}

In our implementation, we maintain two data structures for indexing
processes: a radix tree and a linked list. The radix tree is used to
easily query using the process ID, during the above operation. The
linked list is used during dispatching and will be explained later.

In the \emph{add\_request} function, the pointer stored within the
private field can be used to easily reference the process, which will
be nessesary to set the respective tag values, and insert the request
at the tail of a process specific FIFO queue. During dispatch, SFQ
will submit the request with the lowest start tag (i.e. earliest) to
be serviced. Note that this request is, by definition, at the head of
some process queue. Thus, our implementation iterates through the
linked list of processes, checking the head of each list. Once
finished, it has identified the current miminum, it removes it from
its process queue, and sticks its corresponding SFQ data structure in
a special queue for dispatched requests, and dispatches it.

Upon completion, we remove the completed request from the list of
dispatched requests. We must then check if SFQ's virtual clock needs
to be advanced. Remember, the virtual time of the system is the
minimum start tag of all outstanding requests. Note that, the minumum
start tag must belong to a request which has been dispatched, but not
completed, or be at the head of a process queue. Also note, that if
there are any requests which have been dispatached but not completed,
one of them must hold the minumum start tag. Therefore, we only have
to check if the list of dispatched requests is empty. If it isn't, we
can scan it for the minumum start tag. If it is, we scan the process
queues, exactly as in the dispatch function. 

\subsection{SFQ(D)}

SFQ(D) is identical to SFQ except that SFQ is agnostic of depth. That
is, if the scheduler's dispatch function is invoked, and an
outstanding request is awaiting dispatch, SFQ \emph{will} send a
request. SFQ(D) has a set depth parameter, so, before dispatching a
request, SFQ(D) will check that its depth parameter won't be
violated. To acheive this, a simple counter is maintained. Initialized
to zero, incremented at every dispatch and decremented at every
completion, this counter tells SFQ how many requests are in flight at
any given moment. When dispatch is invoked, SFQ(D) will compare this
value with the depth paramenter and return zero if they are equal.

\subsection{SFQ($D^2$)}

Finally, SFQ($D^2$) is identical to SFQ(D) except that the depth value
is not predetermined at initialization, but rather, adaptively
adjusted during I/O. SFQ($D^2$), instead, begins with a target
lantency for read and write requests, and will adjust the depth value
periotically based on discrepancies between the target and observed
latencies. In order to do this, SFQ($D^2$) must track latency
statistics for each request. When a request is submitted through the
dispatch function, the kernel time is recorded in the request data
structure. Upon completion, the kernel time is again recorded and used
to compute the latency. For each windows, (in our implementation the
window is 1000 requests) we keep a running sum of all request
latencies, the total number of requests for the current window, and
the number of both reads and writes in the current window. At the end
of each windows, we compute the average latency, and a weighted
latency target. The weighted latency target($L_W$) is the read target($T_R$) times
the precentage of requests which were reads plus the write target ($T_{Wr}|$
times the precentage of requests which were writes:

\begin{equation}
  L_W=T_R*(reads/reqs) + T_{Wr}*(writes/reqs)
\end{equation}

Finally, the new depth value ($D_{k+1}$) is set to be the old depth
value $(D_k$) plus the weighted target latency minus the observed
average latency from $kth$ window times some scale value (in our
implementation .03):

\begin{equation}
  D_{k+1}=D_k + S_c*(L_W - AvgLat)
\end{equation}

The dispatch function still checks the depth and returns as in SFQ(D)
except it must consider the new case were the depth is actually
\emph{less} than the current number of dispatched requests, but this
is trivial. After computing and setting depth to $D_{k+1}$, all values
are reset to zero and latency statistics are re-recorded for the next
window.
