\section{Original Project Goals}

The goal of this project is to implement a start-time fair queuing IO
scheduler for flash devices. This scheduler would have a variable I/O
depth attribute($D$), and dynamically calibrate $D$ in order to
achieve and maintain target performance parameters. This scheduler
would be work conserving (so $D$ will always be greater than 1), but the
scheduler should strive to maintain inter-process fairness in terms of
device usage so it may shrink $D$ in order to do so. The name of the
proposed scheduler is SFQ($D^2$) or Start-time Fair Queuing with
Dynamic Depth.

\subsection{Design}

Target performance will be characterized by per request latency and
overall device bandwidth. Fairness with be determined by per process
relative progress. Relative progress is determined by the difference
between the start times of the last issued request from two processes.

SFQ($D^2$) will optimize performance while guaranteeing fairness by
tuning depth using a control loop. This loop will maintain moving
average data about request latency and bandwidth for the last several
requests as well as minimum and maximum observed values. It will also
consider the relative progress of I/O issuing processes. Part of this
project will be to accurately model the relationship between latency,
bandwidth, depth and fairness and implement it in this control
loop. On the surface, latency and fairness should benefit from reduced
depth at the cost of bandwidth and vice versa, but more work is
necessary for the final paper.

\subsection{Evaluation}

For the final paper, we intend in implement SFQ, SFQ(D), FlashFQ, and
SFQ($D^2$) as I/O schedulers on a real Linux system and evaluate them
with a real flash device. Evaluation data would entail performance
benchmarking with per process responsiveness and latency statistics as
well as global device bandwidth. We believe that due to the work
conserving behavior of our scheduler, we will outperform the
aforementioned schedulers in device performance while matching
inter-process relative responsiveness.
