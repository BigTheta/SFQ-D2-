\section{Challenges}

Designing and implementing a scheduler in a matter of days is no easy
task. During the course of the semester, both contributors found
themselves balancing qualifying exams, research
publication/submissions and other coursework which led to considerable
delays in the progress of this project. There is no small irony in the
fact that many lessons on the importance of proper scheduling were
learned over the course of implementing our scheduler.

As for the implementation, we did have several challenges to overcome, including:

1. When benchmarking our device, higher IO depths led to significant
request merging in sequential workloads, resulting in fewer, but
extremely large requests, with high bandwidth and very high latency.
    
To solve this issue, we used random workloads to benchmark the device.

2. Careful consideration must be paid to mutual exclusion within the
kernel. This is doubly true for I/O schedulers. Certain functions can
be invoked in parrallel, resulting in race conditions within critical
shared sections. Certain function are invoked by hardware interrupts,
which are uninterruptable. In this interrupt context, any attempt to
aquire a spinlock may result in a dead lock if that lock is already by
an interrupted process. Therefore, \emph{irq} save and restore
functions are nessesary to prevent deadlocking when mutual exclusion
is nessesary.

3. Memory should never be allocated along the IO path. This includes
calls to \emph{add\_request}. The purpose of the \emph{set\_request}
function is to allow a scheduler to allocate nessesary data structures
before the call to \emph{add\_request}. 

4. Several issues were not reproducable in a virtual context. Early
development was performed on virtual machines, to minimize restart
time after crashes and pipe the crash long through the virtual serial
port to the host. After a short celebration, the scheduler was moved
to a physical system for evaluation. Unfortunately, some race
conditions which were never expressed on a virtual system led to a
second round of debugging without the added efficiency of crash logs
and fast restarts.
